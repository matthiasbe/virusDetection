\documentclass[9.5pt]{beamer}

\mode<presentation>
{
  \usetheme{Madrid}       % or try default, Darmstadt, Warsaw, ...
  \usecolortheme{default} % or try albatross, beaver, crane, ...
  \usefonttheme{serif}    % or try default, structurebold, ...
  \setbeamertemplate{navigation symbols}{}
  \setbeamertemplate{caption}[numbered]
} 

\usepackage[utf8]{inputenc} % accents 8 bits dans le fichier
\usepackage[T1]{fontenc}      % accents codés dans la fonte
\usepackage[french]{babel}
\usepackage{amsmath,amssymb}
\usepackage{graphicx}
\usepackage{fancyhdr}
\usepackage{siunitx}
\usepackage{hepnames}
\usepackage{tikz-feynman}
\usepackage[version=4]{mhchem} 
\usepackage[mode=buildnew]{standalone}
\usepackage{booktabs}
\usepackage{wasysym}
\usepackage{pgfplots}


\DeclareSIUnit\year{yr}

% Here's where the presentation starts, with the info for the title slide
\title[Détection de virus informatique]{Classifications de programmes malicieux et non-malicieux\\ à partir de propriétés binaires}
\author{Matthias \bsc{Beaupère}, Pierre \bsc{Granger}}
\institute{M2 CHPS}
\date{\today}

\begin{document}
\setbeamercolor{captioncolor}{fg=white,bg=red!80!white}
\setbeamertemplate{caption}{%
\begin{beamercolorbox}[wd=0.8\linewidth, sep=.2ex]{captioncolor}\tiny\centering\insertcaption%
\end{beamercolorbox}%
}

\begin{frame}
  \titlepage
\end{frame}

% These three lines create an automatically generated table of contents.


\begin{frame}{Présentation du dataset}
\end{frame}

\begin{frame}{Algorithme : Regression logistique}
\end{frame}

\begin{frame}{Arbre de décision - data-splitting}
	\begin{figure}
		\begin{tikzpicture}
			\begin{axis}[
				width=0.4\linewidth,
				height=6cm,
				ylabel={Moyenne de précision},
				xlabel={Taille de l'ensemble de test}
				],
				\addplot table [x index=0, y index=1, only marks] {../data/decision_tree.txt};
			\end{axis}
		\end{tikzpicture}
		\begin{tikzpicture}
			\begin{axis}[
				width=0.5\linewidth,
				height=6cm,
				ylabel={Ecart-type en précision},
				xlabel={Taille de l'ensemble de test}
				],
				\addplot table [x index=0, y index=2, only marks] {../data/decision_tree.txt};
			\end{axis}
		\end{tikzpicture}
		\caption{Moyenne et Ecart-type pour 1000 data-splitting}
		\label{data-splitting-dt}
	\end{figure}
\end{frame}

\begin{frame}{Arbre de décision - Validation croisée}
\end{frame}

\begin{frame}{Random Forest - Validation croisée}

	\begin{figure}
	\begin{center}
		\begin{tikzpicture}
			\begin{axis}[
				width=0.7\linewidth,
				height=6cm,
				ylabel={Moyenne de précision},
				xlabel={Nombre de parts},
				ymin=0
				],
				\addplot table [x index=0, y index=1, only marks] {../data/random_forest.txt};
			\end{axis}
		\end{tikzpicture}
		\caption{Impact du nombre de division (100 répétitions)}
		\label{random_forest}
	\end{center}
	\end{figure}
\end{frame}

\begin{frame}{Random Forest - Impact du nombre d'arbres}
	\begin{figure}
	\begin{center}
		\begin{tikzpicture}
			\begin{axis}[
				width=0.7\linewidth,
				height=6cm,
				ylabel={Moyenne de précision},
				xlabel={Nombre d'estimateurs}
				],
				\addplot table [x index=0, y index=1, only marks] {../data/random_forest_estimators.txt};
			\end{axis}
		\end{tikzpicture}
		\caption{Impact du nombre d'estimateurs (100 répétitions)}
		\label{rf_estimators}
	\end{center}
	\end{figure}
\end{frame}


\begin{frame}{Random Forest - Impact de la profondeur}
	\begin{figure}
		\begin{center}
			\begin{tikzpicture}
				\begin{axis}[
					width=0.7\linewidth,
					height=6cm,
					ylabel={Moyenne de précision},
					xlabel={Profondeur maximum d'arbre}
					],
					\addplot table [x index=0, y index=1, only marks] {../data/random_forest_depth.txt};
				\end{axis}
			\end{tikzpicture}
			\caption{Impact de la profondeur maximale (100 répétitions)}
			\label{rf_depth}
		\end{center}
	\end{figure}
\end{frame}

\begin{frame}{Analyse des résultats}
\end{frame}

\begin{frame}{Conclusion}
\end{frame}


\end{document}
